\documentclass[12pt, spanish]{article}
\usepackage[spanish]{babel}
\selectlanguage{spanish}
%\usepackage{natbib}
\usepackage{url}
\usepackage[utf8x]{inputenc}
\usepackage{graphicx}
\graphicspath{{images/}}
\usepackage{parskip}
\usepackage{fancyhdr}
\usepackage{vmargin}
\usepackage{multirow}
\usepackage{float}
\usepackage{chngpage}

\usepackage{amsfonts}

\usepackage{subcaption}

\usepackage{hyperref}
\usepackage[
    type={CC},
    modifier={by-nc-sa},
    version={4.0},
]{doclicense}

\hypersetup{
    colorlinks=true,
    linkcolor=blue,
    filecolor=magenta,
    urlcolor=cyan,
}

% para codigo
\usepackage{listings}
\usepackage{xcolor}



%% configuración de listings

\definecolor{listing-background}{HTML}{F7F7F7}
\definecolor{listing-rule}{HTML}{B3B2B3}
\definecolor{listing-numbers}{HTML}{B3B2B3}
\definecolor{listing-text-color}{HTML}{000000}
\definecolor{listing-keyword}{HTML}{435489}
\definecolor{listing-identifier}{HTML}{435489}
\definecolor{listing-string}{HTML}{00999A}
\definecolor{listing-comment}{HTML}{8E8E8E}
\definecolor{listing-javadoc-comment}{HTML}{006CA9}

\lstdefinestyle{eisvogel_listing_style}{
  language         = python,
%$if(listings-disable-line-numbers)$
%  xleftmargin      = 0.6em,
%  framexleftmargin = 0.4em,
%$else$
  numbers          = left,
  xleftmargin      = 0em,
 framexleftmargin = 0em,
%$endif$
  backgroundcolor  = \color{listing-background},
  basicstyle       = \color{listing-text-color}\small\ttfamily{}\linespread{1.15}, % print whole listing small
  breaklines       = true,
  frame            = single,
  framesep         = 0.19em,
  rulecolor        = \color{listing-rule},
  frameround       = ffff,
  tabsize          = 4,
  numberstyle      = \color{listing-numbers},
  aboveskip        = 1.0em,
  belowskip        = 0.1em,
  abovecaptionskip = 0em,
  belowcaptionskip = 1.0em,
  keywordstyle     = \color{listing-keyword}\bfseries,
  classoffset      = 0,
  sensitive        = true,
  identifierstyle  = \color{listing-identifier},
  commentstyle     = \color{listing-comment},
  morecomment      = [s][\color{listing-javadoc-comment}]{/**}{*/},
  stringstyle      = \color{listing-string},
  showstringspaces = false,
  escapeinside     = {/*@}{@*/}, % Allow LaTeX inside these special comments
  literate         =
  {á}{{\'a}}1 {é}{{\'e}}1 {í}{{\'i}}1 {ó}{{\'o}}1 {ú}{{\'u}}1
  {Á}{{\'A}}1 {É}{{\'E}}1 {Í}{{\'I}}1 {Ó}{{\'O}}1 {Ú}{{\'U}}1
  {à}{{\`a}}1 {è}{{\'e}}1 {ì}{{\`i}}1 {ò}{{\`o}}1 {ù}{{\`u}}1
  {À}{{\`A}}1 {È}{{\'E}}1 {Ì}{{\`I}}1 {Ò}{{\`O}}1 {Ù}{{\`U}}1
  {ä}{{\"a}}1 {ë}{{\"e}}1 {ï}{{\"i}}1 {ö}{{\"o}}1 {ü}{{\"u}}1
  {Ä}{{\"A}}1 {Ë}{{\"E}}1 {Ï}{{\"I}}1 {Ö}{{\"O}}1 {Ü}{{\"U}}1
  {â}{{\^a}}1 {ê}{{\^e}}1 {î}{{\^i}}1 {ô}{{\^o}}1 {û}{{\^u}}1
  {Â}{{\^A}}1 {Ê}{{\^E}}1 {Î}{{\^I}}1 {Ô}{{\^O}}1 {Û}{{\^U}}1
  {œ}{{\oe}}1 {Œ}{{\OE}}1 {æ}{{\ae}}1 {Æ}{{\AE}}1 {ß}{{\ss}}1
  {ç}{{\c c}}1 {Ç}{{\c C}}1 {ø}{{\o}}1 {å}{{\r a}}1 {Å}{{\r A}}1
  {€}{{\EUR}}1 {£}{{\pounds}}1 {«}{{\guillemotleft}}1
  {»}{{\guillemotright}}1 {ñ}{{\~n}}1 {Ñ}{{\~N}}1 {¿}{{?`}}1
  {…}{{\ldots}}1 {≥}{{>=}}1 {≤}{{<=}}1 {„}{{\glqq}}1 {“}{{\grqq}}1
  {”}{{''}}1
}
\lstset{style=eisvogel_listing_style}


\usepackage[default]{sourcesanspro}

\setmarginsrb{2 cm}{1 cm}{2 cm}{2 cm}{1 cm}{1.5 cm}{1 cm}{1.5 cm}

\title{Práctica 1:\\
Filtros de máscaras  \hspace{0.05cm} }
\author{Antonio David Villegas Yeguas}
\date{\today}

\renewcommand*\contentsname{hola}

\makeatletter
\let\thetitle\@title
\let\theauthor\@author
\let\thedate\@date
\makeatother

\pagestyle{fancy}
\fancyhf{}
\rhead{\theauthor}
\lhead{\thetitle}
\cfoot{\thepage}

\begin{document}

%%%%%%%%%%%%%%%%%%%%%%%%%%%%%%%%%%%%%%%%%%%%%%%%%%%%%%%%%%%%%%%%%%%%%%%%%%%%%%%%%%%%%%%%%

\begin{titlepage}
    \centering
    \vspace*{0.3 cm}
    \includegraphics[scale = 0.50]{ugr.png}\\[0.7 cm]
    %\textsc{\LARGE Universidad de Granada}\\[2.0 cm]
    \textsc{\large 4º CSI 2020/21 - Grupo 2}\\[0.5 cm]
    \textsc{\large Grado en Ingeniería Informática}\\[0.5 cm]
    \rule{\linewidth}{0.2 mm} \\[0.2 cm]
    { \huge \bfseries \thetitle}\\
    \rule{\linewidth}{0.2 mm} \\[1 cm]

    \begin{minipage}{0.4\textwidth}
        \begin{flushleft} \large
            \emph{Autor:}\\
            \theauthor\\
			 \emph{DNI:}\\
            77021623-M
            \end{flushleft}
            \end{minipage}~
            \begin{minipage}{0.4\textwidth}
            \begin{flushright} \large
            \emph{Asignatura: \\
            Visión por Computador}   \\
            \emph{Correo:}\\
            advy99@correo.ugr.es
        \end{flushright}
    \end{minipage}\\[0.5cm]

    {\large \thedate}\\[0.5cm]
    %{\url{https://github.com/advy99/AA/}}
    {\doclicenseThis}

    \vfill

\end{titlepage}

%%%%%%%%%%%%%%%%%%%%%%%%%%%%%%%%%%%%%%%%%%%%%%%%%%%%%%%%%%%%%%%%%%%%%%%%%%%%%%%%%%%%%%%%%

\tableofcontents
\pagebreak

%%%%%%%%%%%%%%%%%%%%%%%%%%%%%%%%%%%%%%%%%%%%%%%%%%%%%%%%%%%%%%%%%%%%%%%%%%%%%%%%%%%%%%%%%


\section*{Introducción}

Para esta primera práctica trabajaremos los filtros de máscaras, con el objetivo de mostrar como usando técnicas de filtrado lineal es posible extraer información relevante de una imagen para su posterior interpretación.

En esta práctica también nos centraremos en comparar nuestra implementación propia con la implementación de la biblioteca OpenCV. Por este motivo, en todos los ejercicios utilizaré funciones implementadas en el fichero de Python entregado y en ciertos momentos utilizaré las funciones de filtros que nos ofrece OpenCV pero solo de forma comparativa.

\section{Ejercicio 1: Máscaras}

Este ejercicio trata sobre el cálculo de las máscaras Gaussianas que utilizaremos durante toda la práctica, la aplicación de dichas máscaras a una imagen en forma de una convolución 2D y el cálculo de máscaras normalizadas de la Laplaciana de la Gaussiana con nuestras propias máscaras.

Además de la implementación de todas estas máscaras, también realizaremos una comparación con las operaciones equivalentes de OpenCV.

\subsection{Máscaras discretas 1D}

De cara a obtener las distintas máscaras discretas he implementado tres funciones de Python:

\begin{enumerate}
	\item Función Gaussiana: Devuelve el valor de la función gaussiana en un punto dado, para un sigma dado.
	\item Función derivada de la Gaussiana: Devuelve el valor de la derivada de la gaussiana en un punto dado, para un sigma dado.
	\item Función segunda derivada de la Gaussiana: Devuelve el valor de la segunda derivada de la gaussiana en un punto dado, para un sigma dado.
\end{enumerate}


Para el cálculo de las distintas funciones simplemente he evaluado la función gaussiana, obviando la constante $c$ como hemos visto en teoría y como nos comentaron en prácticas:

\begin{figure}[H]
	\centering
	\[ f(x) = c \cdot e^{- \frac{x^2}{2\sigma^2}} \]
	\caption{Función gaussiana.}
	\label{f_gaussiana}
\end{figure}

\newpage

De cara a las funciones derivadas, simplemente he derivado la función gaussiana respecto de $x$:


\begin{figure}[H]
	\centering
	\[ \frac{\partial{f(x)}}{\partial{x}} = -\frac{e^{ -\frac{x^2}{2 \cdot \sigma^2} } \cdot x}{\sigma^2} \]
	\caption{Derivada de la función gaussiana.}
	\label{d_f_gaussiana}
\end{figure}

\begin{figure}[H]
	\centering
	\[ \frac{\partial{f(x)}}{\partial{x^2}} = -\frac{- x^2 + \sigma^2}{e^{ \frac{x^2}{2 \cdot \sigma^2} } \cdot \sigma^4} \]
	\caption{Segunda derivada de la función gaussiana.}
	\label{2d_f_gaussiana}
\end{figure}


Tras implementar estas funciones, que simplemente evaluavan una función matemática, he implementado la función \texttt{kernel\_gaussiano\_1d}, que obligatoriamente ha de recibir, o bien un sigma, o un tamaño para la máscara. Opcionalmenten puede recibir sobre que función se va a calcular la máscara, por defecto la función gaussiana, de forma que con esta función podamos calcular la máscara gaussiana, la máscara de la derivada de la gaussiana si le pasamos dicha función y la de la segunda derivada de la gaussiana si le pasamos esta última.

Como he comentado, esta función ha de recibir un sigma o un tamaño de máscara. Si recibe un sigma, calcula el tamaño de la máscara y devuelve la máscara evaluando los valores desde el entero menor a $\frac{tam\_mascara}{2}$ hasta $\frac{tam\_mascara}{2} + 1$ con el sigma dado. Si recibe el tamaño de máscara, calculará el sigma y devolverá la máscara calculada de la misma forma. Si no recibimos ni un sigma ni un tamaño de máscara, devolverá una máscara vacia.

Además, como vimos en teoría, si calculamos la máscara utilizando la función gaussiana antes de devolver el resultado lo noralizaremos dividiendo cada elemento de la máscara por la suma de todos, de forma que la máscara sume uno.

De cara a calcular el sigma o el tamaño de máscara seguiremos esta fórmula:


\begin{figure}[H]
	\centering
	\[ 2 \cdot 3 \cdot \sigma + 1 = T\]
	\caption{Calculo de sigma o del tamaño de máscara.}
	\label{sigma_t}
\end{figure}

Siendo T el tamaño de la máscara, podemos calcular dicho T o sigma en caso de que nos proporcionen T. Utilizamos esta fórmula ya que nos asegura que el tamaño de la máscara es al menos $3 \cdot \sigma$ como nos se nos recomendaba en teoría.

Otro detalle a observar es que si nos dan un sigma, el tamaño de la máscara es impar ya que, como también nos enseñaron en teoría, el uso de máscaras pares es problemático ya que no podemos centrar la máscara en un punto concreto de la imagen.

Tras realizar distintas ejecuciones de cara a probar el funcionamiento de esta función obtenemos lo siguiente:

\begin{lstlisting}
kernel_gaussiano_1d(sigma=1)
[0.00443305 0.05400558 0.24203623 0.39905028 0.24203623 0.05400558 0.00443305]

kernel_gaussiano_1d(func=derivada_f_gaussiana, tam_mascara=5)
[0.04999048442209038, 0.7304680515562869, -0.0, -0.7304680515562869, -0.04999048442209038]

kernel_gaussiano_1d(func=derivada_f_gaussiana, tam_mascara=7)
[0.033326989614726917, 0.2706705664732254, 0.6065306597126334, -0.0, -0.6065306597126334, -0.2706705664732254, -0.033326989614726917]
\end{lstlisting}


Podemos comprobar que obtenemos los resultados esperados, ya que el cálculo utilizando la función gaussiana hace que la máscara sume uno, y tanto la primera como segunda derivada suman cero, como era de esperar al haber estudiado estas funciones de forma teórica.


\subsection{Convolución gaussiana 2D}

Para desarrollar este ejercicio he creado la función \texttt{aplicar\_convolucion}, que recibe como parámetros la imagen, la máscara a aplicar de forma horizontal, la máscara a aplicar de forma vertical y el tipo de borde a añadir.

Este ultimo parámetro lo utilizaremos de cara a añadir bordes extra a la imagen, ya que si no aplicamos esta operación parte de los pixeles originales de las imágenes no se les aplicarían las máscaras, ya que el propio tamaño de máscara no permite centrarlas en dichos píxeles sin salir de la imágen.


Esta función sigue los siguientes pasos:

\begin{enumerate}
	\item Aplicar borde: Aplica un borde de tamaño $(len(mascara) - 1) / 2$ a la imagen utilizando la función de OpenCV \texttt{copyMakeBorder}.
	\item Crea una matriz rellena de ceros del tamaño de la imagen original (sin bordes), donde almacenaremos la imagen final.
	\item Recorre la imágen con borde (matriz) desde el tamaño del borde hasta el extremo de la imagen menos los bordes, de forma que pasamos por todos los píxeles de la imagen original. En este recorrido aplica la máscara horizontal, multiplicando la máscara por los pixeles de una misma fila, pero tantos como tamaño de máscara, centrando la máscara en el pixel actual que estamos recorriendo. De esta forma se aplica la máscara en el pixel que queremos y utilizando los bordes añadidos para los valores de la máscara que en principio quedarían fuera de la imagen original. Almacena el resultado en la matriz creada en el paso 2.
	\item Tras aplicar la máscara horizontal, volvemos a añadir los bordes a la imagen con la máscara aplicada, de cara a aplicar la máscara vertical.
	\item Repetimos el paso 3, pero al multiplicar la máscara, en lugar de mantener la fila y coger pixeles de distintas columnas, mantenemos la columna y escogemos un rango de píxeles en las filas para aplicar la máscara vertical. Almacenamos el resultado en la matriz creada en el paso 2.
\end{enumerate}

Tras esta explicación de como funciona la función cabe mencionar distintos detalles de implementación.

Para empezar, vemos como aplico las máscaras de forma separada, y no como una convolución 2D. Esta decisión es tomada ya que el resultado obtenido es el mismo, sin embargo son necesarias menos operaciones para realizar el cálculo gracias a la separabilidad de las convoluciones 2D, haciendo más eficiente esta forma de aplicar las máscaras\cite{separabilidad_2D}.

Otro detalle es el hecho de añadir los bordes cada vez que aplico una convolución 1D. Aunque de esta forma es necesario mayor tiempo de cómputo al tener que volver a añadir bordes, se obtiene un mejor resultados en los bordes de la imagen resultante ya que si solamente se añaden bordes al inicio, al aplicar la segunda convolución usaría los bordes de la imagen original, que si utilizamos una técnica de replicado de bordes son distintos a los bordes de la imagen con una máscara aplicada, haciendo que al aplicar la segunda máscara en las zonas cercanas a los bordes el resultado no sea tan exacto.


Finalmente, obtenemos este resultado aplicando una máscara de tamaño 9 y comparandola con GaussianBlur, con un tamaño de máscara 9 es el siguiente:


\begin{figure}[H]
  \centering
      \includegraphics[width=\textwidth]{ejercicio1-b.png}
 		 \caption{Comparación entre implementación propia y GaussianBlur}
  		\label{fig:ej1b}

\end{figure}

Observamos como el resultado es similar, por lo que intuimos que la implementación tanto de la forma de obtener la máscara y la aplicación de la convolución es correcta.

\subsection{Comparación de máscaras 1D, implementación propia y OpenCV}

De cara a comparar mi implementación con OpenCV, como nos pide el ejercicio, he generado las máscaras 1D con distintos tamaños de máscara y sigma y las he dibujado utilizando MatPlotLib:


\begin{figure}[H]
  \centering
	\begin{subfigure}[t]{0.4\textwidth}
		\centering
		\includegraphics[width = \textwidth]{cmp-p5.png}
 		 \caption{Máscara primera derivada propia, t. máscara 5}
	\end{subfigure}
	\hspace{1cm}
	\begin{subfigure}[t]{0.4\textwidth}
		\centering
		\includegraphics[width = \textwidth]{cmp-cv5.png}
 		 \caption{Máscara primera derivada OpenCV, t. máscara 5}
	\end{subfigure}
	\caption{Comparación primera derivada, tamaño de máscara 5}
  	\label{fig:ej1c5}
\end{figure}


\begin{figure}[H]
  \centering
	\begin{subfigure}[t]{0.4\textwidth}
		\centering
		\includegraphics[width = \textwidth]{cmp-2p5.png}
 		 \caption{Máscara segunda derivada propia, t. máscara 5}
	\end{subfigure}
	\hspace{1cm}
	\begin{subfigure}[t]{0.4\textwidth}
		\centering
		\includegraphics[width = \textwidth]{cmp-2cv5.png}
 		 \caption{Máscara segunda derivada OpenCV, t. máscara 5}
	\end{subfigure}
	\caption{Comparación segunda derivada, tamaño de máscara 5}

  	\label{fig:ej1c5}
\end{figure}


\begin{figure}[H]
  \centering
	\begin{subfigure}[t]{0.4\textwidth}
		\centering
		\includegraphics[width = \textwidth]{cmp-p9.png}
 		 \caption{Máscara primera derivada propia, t. máscara 9}
	\end{subfigure}
	\hspace{1cm}
	\begin{subfigure}[t]{0.4\textwidth}
		\centering
		\includegraphics[width = \textwidth]{cmp-cv9.png}
 		 \caption{Máscara primera derivada OpenCV, t. máscara 9}
	\end{subfigure}
	\caption{Comparación primera derivada, tamaño de máscara 9}
  	\label{fig:ej1c5}
\end{figure}


\begin{figure}[H]
  \centering
	\begin{subfigure}[t]{0.4\textwidth}
		\centering
		\includegraphics[width = \textwidth]{cmp-2p9.png}
 		 \caption{Máscara segunda derivada propia, t. máscara 9}
	\end{subfigure}
	\hspace{1cm}
	\begin{subfigure}[t]{0.4\textwidth}
		\centering
		\includegraphics[width = \textwidth]{cmp-2cv9.png}
 		 \caption{Máscara segunda derivada OpenCV, t. máscara 9}
	\end{subfigure}
	\caption{Comparación segunda derivada, tamaño de máscara 9}

  	\label{fig:ej1c5}
\end{figure}


\begin{figure}[H]
  \centering
	\begin{subfigure}[t]{0.4\textwidth}
		\centering
		\includegraphics[width = \textwidth]{cmp-p13.png}
 		 \caption{Máscara primera derivada propia, t. máscara 13}
	\end{subfigure}
	\hspace{1cm}
	\begin{subfigure}[t]{0.4\textwidth}
		\centering
		\includegraphics[width = \textwidth]{cmp-cv13.png}
 		 \caption{Máscara primera derivada OpenCV, t. máscara 13}
	\end{subfigure}
	\caption{Comparación primera derivada, tamaño de máscara 13}
  	\label{fig:ej1c5}
\end{figure}


\begin{figure}[H]
  \centering
	\begin{subfigure}[t]{0.4\textwidth}
		\centering
		\includegraphics[width = \textwidth]{cmp-2p13.png}
 		 \caption{Máscara segunda derivada propia, t. máscara 13}
	\end{subfigure}
	\hspace{1cm}
	\begin{subfigure}[t]{0.4\textwidth}
		\centering
		\includegraphics[width = \textwidth]{cmp-2cv13.png}
 		 \caption{Máscara segunda derivada OpenCV, t. máscara 13}
	\end{subfigure}
	\caption{Comparación segunda derivada, tamaño de máscara 13}

  	\label{fig:ej1c5}
\end{figure}



\begin{figure}[H]
  \centering
	\begin{subfigure}[t]{0.4\textwidth}
		\centering
		\includegraphics[width = \textwidth]{cmp-p17.png}
 		 \caption{Máscara primera derivada propia, t. máscara 17}
	\end{subfigure}
	\hspace{1cm}
	\begin{subfigure}[t]{0.4\textwidth}
		\centering
		\includegraphics[width = \textwidth]{cmp-cv17.png}
 		 \caption{Máscara primera derivada OpenCV, t. máscara 17}
	\end{subfigure}
	\caption{Comparación primera derivada, tamaño de máscara 17}
  	\label{fig:ej1c5}
\end{figure}


\begin{figure}[H]
  \centering
	\begin{subfigure}[t]{0.4\textwidth}
		\centering
		\includegraphics[width = \textwidth]{cmp-2p17.png}
 		 \caption{Máscara segunda derivada propia, t. máscara 17}
	\end{subfigure}
	\hspace{1cm}
	\begin{subfigure}[t]{0.4\textwidth}
		\centering
		\includegraphics[width = \textwidth]{cmp-2cv17.png}
 		 \caption{Máscara segunda derivada OpenCV, t. máscara 17}
	\end{subfigure}
	\caption{Comparación segunda derivada, tamaño de máscara 17}

  	\label{fig:ej1c5}
\end{figure}


\begin{figure}[H]
  \centering
	\begin{subfigure}[t]{0.4\textwidth}
		\centering
		\includegraphics[width = \textwidth]{cmp-p21.png}
 		 \caption{Máscara primera derivada propia, t. máscara 21}
	\end{subfigure}
	\hspace{1cm}
	\begin{subfigure}[t]{0.4\textwidth}
		\centering
		\includegraphics[width = \textwidth]{cmp-cv21.png}
 		 \caption{Máscara primera derivada OpenCV, t. máscara 21}
	\end{subfigure}
	\caption{Comparación primera derivada, tamaño de máscara 21}
  	\label{fig:ej1c5}
\end{figure}


\begin{figure}[H]
  \centering
	\begin{subfigure}[t]{0.4\textwidth}
		\centering
		\includegraphics[width = \textwidth]{cmp-2p21.png}
 		 \caption{Máscara segunda derivada propia, t. máscara 21}
	\end{subfigure}
	\hspace{1cm}
	\begin{subfigure}[t]{0.4\textwidth}
		\centering
		\includegraphics[width = \textwidth]{cmp-2cv21.png}
 		 \caption{Máscara segunda derivada OpenCV, t. máscara 21}
	\end{subfigure}
	\caption{Comparación segunda derivada, tamaño de máscara 21}

  	\label{fig:ej1c5}
\end{figure}


\subsection{Máscaras Laplacianas}



\section{Ejercicio 2: Pirámides}

\subsection{Pirámide Gaussiana}

\subsection{Pirámide Laplaciana}



\section{Ejercicio 3: Imágenes híbridas}



\section{Bonus}

\newpage

\section{Referencias, material y documentación usada}


\begin{thebibliography}{9}

\bibitem{separabilidad_2D}
	Prueba de separabilidad para las convoluciones 2D.

	\url{https://www.songho.ca/dsp/convolution/convolution2d_separable.html}

\end{thebibliography}

\end{document}
